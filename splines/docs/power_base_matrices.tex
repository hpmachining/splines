\documentclass[10pt,letterpaper]{article}
\usepackage[utf8]{inputenc}
\usepackage{amsmath}
\usepackage{amsfonts}
\usepackage{amssymb}
\makeatletter
\renewcommand*\env@matrix[1][c]{\hskip -\arraycolsep
  \let\@ifnextchar\new@ifnextchar
  \array{*\c@MaxMatrixCols #1}}
\makeatother
\begin{document}
\noindent
The $n + 1$ Bernstein basis polynomials of degree $n$ are defined as
\[
 b_{v,n}(t) =  \left(
\begin{matrix}
n\\
v
\end{matrix}
\right)x^t(1 - t)^{n - v}, v = 0,...,n
\]
\\
The coefficients $\beta_v$ are the Bernstein or Bézier coefficients.
A linear combination of Bernstein basis polynomials
\[
B_n(t) = \displaystyle\sum_{v = 0}^n\beta_v b_{v,n}(t)
\]
\begin{align*}
B(t) &= 1 \\
B(t) &= (1-t)^2 + 2t(1-t) + t^2 \\
B(t) &= (1-t)^3 + 3t(1-t)^2 + 3t^2(1-t) + t^3 \\
B(t) &= (1-t)^4 + 4t(1-t)^3 + 6t^2(1-t)^3 + 4t^3(1-t) + t^4 \\
\end{align*}
Control Points given as $ \lbrace P_1, P_2, P_3, ..., P_{n+1}\rbrace$\\
\\
Matrix form for Quadratic Bézier curves (degree 2)\\
\[
B(t) =
\begin{bmatrix} 1 & t & t^2 \end{bmatrix}
\cdot
\begin{bmatrix}[r]
1 & 0 & 0\\
-2 & 2 & 0\\
1 & -2 & 1
\end{bmatrix}
\cdot
\begin{bmatrix}
P_1\\
P_2\\
P_3
\end{bmatrix}
\]
\newline
\newline
Cubic Bézier curves (degree 3)
\[
B(t) =
\begin{bmatrix}
1 & t & t^2 & t^3
\end{bmatrix}
\cdot
\begin{bmatrix}[r]
1 & 0 & 0 & 0\\
-3 & 3 & 0 & 0\\
3 & -6 & 3 & 0\\
-1 & 3 & -3 & 1
\end{bmatrix}
\cdot
\begin{bmatrix}
P_1\\
P_2\\
P_3\\
P_4
\end{bmatrix}
\]
\newline
\newline
Quartic Bézier curves (degree 4)
\[
B(t) =
\begin{bmatrix} 1 & t & t^2 & t^3 & t^4 \end{bmatrix}
\cdot
\begin{bmatrix}[r]
1 & 0 & 0 & 0 & 0\\
-4 & 4 & 0 & 0 & 0\\
6 & -12 & 6 & 0 & 0\\
-4 & 12 & -12 & 4 & 0\\
1 & -4 & 6 & -4 & 1
\end{bmatrix}
\cdot
\begin{bmatrix}
P_1\\
P_2\\
P_3\\
P_4\\
P_5
\end{bmatrix}
\]
\newline
\newline
Quintic Bézier curves (degree 5)
\[
B(t) =
\begin{bmatrix} 1 & t & t^2 & t^3 & t^4 & t^5 \end{bmatrix}
\cdot
\begin{bmatrix}[r]
1 & 0 & 0 & 0 & 0 & 0\\
-5 & 5 & 0 & 0 & 0 & 0\\
10 & -20 & 10 & 0 & 0 & 0\\
-10 & 30 & -30 & 10 & 0 & 0\\
5 & -20 & 30 & -20 & 5 & 0\\
-1 & 5 & -10 & 10 & -5 & 1
\end{bmatrix}
\cdot
\begin{bmatrix}
P_1\\
P_2\\
P_3\\
P_4\\
P_5\\
P_6
\end{bmatrix}
\]
\newline

\end{document}